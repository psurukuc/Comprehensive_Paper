\documentclass[11pt]{article}
\usepackage{graphicx,amsmath,amsthm,amssymb,amsfonts}
\usepackage{hyperref}
\usepackage{xcolor}
\usepackage{datetime}
\hypersetup{
    colorlinks,
    linkcolor={red!50!black},
    citecolor={blue!50!black},
    urlcolor={blue!80!black}
}
\usepackage{cite}
\author{Pranava Teja Surukuchi \\ Illinois Institute of Technology\\ Chicago, IL 60616}
\title{Measurement of reactor flux at short baseline in search for possible oscillations arising from the existence of eV-scale sterile neutrinos}
\newdate{date}{30}{04}{2015}
\date{\displaydate{date}}
\setlength{\topmargin}{-0.5 in}
\setlength{\textheight}{8.8 in}
\addtolength{\oddsidemargin}{-.875in}
\addtolength{\evensidemargin}{-.875in}
\addtolength{\textwidth}{1.75in}
\begin{document}
\maketitle
\begin{center}
\textit{(in partial fulfillment of comprehensive exam for Doctor of Philosophy in Physics)}
\end{center}
\newpage

\tableofcontents

\section{Neutrinos Background and theory}

\subsection{History}

\subsection{phenomenology}

\subsection{Establishment of certain properties in Neutrinos}

\subsection{Reactor Neutrino experiments}

\section{Anomalies}

\subsection{Reactor Antineutrino anomaly }

\subsection{all others}

\subsection{blahblah}

\subsection{Possible reasons for anomalies}

\section{PROSPECT}

\cite{brierley2011amplitude}

\subsection{Location selection etc}

\subsection{Background}

\subsection{Detector characterisitcs}

\section{Proposed PhD research}

\section{Current work}

\bibliographystyle{plain}	
\bibliography{References}

\end{document}
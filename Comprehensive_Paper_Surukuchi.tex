% Font size 11 and article type.
\documentclass[11pt]{article}
% For managing pictures, making various arrangements of equations easier to use and also makes them prettier
% Can use ams for creating own matjematical operators using \DeclareMathOperator (Also check \newcommand)
\usepackage{graphicx,amsmath,amsthm,amssymb,amsfonts}
%For references
\usepackage{hyperref}
% For managing colors
\usepackage{xcolor}
%Date time management 
\usepackage{datetime}
% Package for citing, reduces several authors to et al.
\usepackage{hepparticles}
% Setup for color displays of hyperlinks, change the values to get different colors.
\hypersetup{
    colorlinks,
    linkcolor={red!50!black},
    citecolor={blue!50!black},
    urlcolor={blue!80!black}
}
% Package for citing, not needed as long as we have heparticles package, I think.
%\usepackage{cite}
\author{Pranava Teja Surukuchi \\ Illinois Institute of Technology\\ Chicago, IL 60616}
% Needs to be changed sounds phony
\title{Measurement of reactor flux at short baseline in search for possible oscillations arising from the existence of eV-scale sterile neutrinos}
\newdate{date}{30}{04}{2015}
\date{\displaydate{date}}
\setlength{\topmargin}{-0.5 in}
\setlength{\textheight}{8.8 in}
\addtolength{\oddsidemargin}{-0.5in}
\addtolength{\evensidemargin}{-.875in}
\addtolength{\textwidth}{1in}
\begin{document}
\maketitle
\begin{center}
% Not sure about this either
\textit{(in partial fulfillment of comprehensive exam for Doctor of Philosophy in Physics)}
\end{center}
\newpage

\tableofcontents

\newpage


\section{Neutrino background and theory}
Neutrinos are neutral, very light, weakly interacting subatomic particle that make up the universe. They are second highly abundant fundamental particles in universe, but since they are very light and they can only interact via weak force and gravity, they remained obscure for a long time. Even after six decades of their discovery, fascinating properties of neutrinos are uncovered time after time. Neutrino oscillation, one such property arises from the fact that the flavor eigenstates of neutrinos ( $ \nu_{e}, \nu_{\mu}, \nu_{\tau} $ ) do not coincide with their mass eigenstates ( $ \nu_{1}, \nu_{2}, \nu_{3} $ ). The increasing sensitivity of measuring devices coupled with improving experimental techniques led to very precise understanding of the parameters of neutrino oscillations (cite). However, there are still several compelling questions regarding neutrinos that haven't been answered . 

\subsection{Postulation and discovery of neutrino}
\label{discovery}
History of neutrinos can be traced back to the discovery of radioactivity. After the discovery of radioactive decay, radioactive displacement law was formulated by Soddy and Fajans(cite) to described the products of alpha and beta decays. Measurements of beta decay spectrum in several experiments including measurement of spectrum of beta decay by James Chadwick show that it is continuous. Further the spin of Nitrogen, in contradiction to Rutherford's prediction(cite), turned out to be one (cite). To explain these discrepancies, Wolfgang Pauli(cite) in 1931, postulated existence of a subatomic particle, which he called `neutron'. Fermi's theory of Beta decay added further strong theoretical foundation to this particle. Because an uncharged particle with the name `neutron' already exists, Fermi in his theory named it `neutrino'.

It was not until 1956 that neutrino was experimentally discovered by Reines and Cowan at Savannah river plant in North Carolina.  Anti-neutrinos from the Savannah river reactor are captured by inverse beta decay on protons in water. 

$$ \bar{\nu_{e}} + p \rightarrow e^{+} + n  $$
The positron from inverse beta decay annihilates with electrons promptly and generates two gammas.
$$ e^{+} + e^{-} -\rightarrow \gamma + \gamma $$
Gammas thus generated are lowered in wavelength by the use of wavelength shifters and are collected by photo-multiplier tubes. The neutron from inverse beta decay, after thermalizing captures on cadmium. 
$$ n + ^{108}Cd \rightarrow \gamma + ^ {109}Cd  $$ 
A delayed coincidence between the proton and the neutron has been established. The rector was shut down and reactor-on versus reactor-off data was compared to to conclusively prove that the events are indeed from inverse beta decay. Most of the reactor experiments till date are performed along the same lines utilizing inverse beta decay. 
\subsection{Uncovering of neutrino properties}
Not sure if needed
\subsection{Role of reactors in neutrino experiments}
Reactors played very important role in neutrino experiments starting from its discovery at Savannah river(cite ) discussed above \ref{discovery} to discovery of the oscillation parameter $ \theta_{13} $ in Dayabay (cite). 

\section{Anomalies and potential for new physics}

\subsection{Reactor Antineutrino anomaly }

\subsection{Various other observational discrepancies}

\subsection{Possible explanations for observed anomalies}

\section{Phenomenology}

\subsection{Neutrino and weak interactions(questionable if needed)}

\subsection{Neutrino Oscillations}

\subsection{Sterile neutrino }

\section{PROSPECT experiment}

\cite{2013arXiv1309.7647A}

\subsection{Physics Goals}

\subsection{Reactor location and backgrounds}

\subsection{Detector Design}

\section{Proposed PhD research}

\subsection{Oscillation measurement}

\subsection{Timeline}

\section{Current and past work}

\subsection{Characterization of detector material}

\subsection{Investigation of structural and mechanical properties of segmentation design}

\subsection{Identification of PSD method}

\subsection{Covariance matrix method for identification of effects of varying properties of the detector}



\bibliographystyle{h-physrev}
\bibliography{References}

\end{document}